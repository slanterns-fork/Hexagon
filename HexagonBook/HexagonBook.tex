% !TEX encoding = UTF-8
% !TEX program = xelatex
\documentclass{tufte-book}
\usepackage[CJKchecksingle]{xeCJK}
\usepackage{CJKnumb}
\usepackage{indentfirst}
\usepackage{minitoc}
\usepackage[utf8]{inputenc}
\usepackage{xunicode}
\usepackage{xltxtra}
\renewcommand{\partname}{}
\renewcommand{\thepart}{第\CJKnumber{\arabic{part}}部分}

\usepackage{makeidx}
\makeindex


\renewcommand{\tablename}{表}
\renewcommand{\figurename}{图}
\renewcommand{\contentsname}{目录}
\renewcommand{\indexname}{索引}
\renewcommand{\listfigurename}{插图目录}
\renewcommand{\listtablename}{表格目录}




\begin{document}
\setlength{\parindent}{2em}
\setcounter{tocdepth}{1}

\frontmatter

\title{Hexagon~编程指南}
\author{0xHexadecimal}
\publisher{内部文档}
\maketitle

\chapter{概览}
Hexagon~是一门全新的编程语言。它具有全新的完美类库分离语法、完全函数编程支持、强面向集合性等诸多令人喜爱的特性以及一些经过改善的语法。一睹为快吧。
\section{Hello, world!}
%\usepackage{graphics} is needed for \includegraphics
\begin{figure}[htp]
\begin{center}
  	\begin{verbatim}
		void() Main = (){
		    var c = import Hexagon::System::Console;
		    c << "Hello, World!" << Hexagon::System::SpecChar::EndLine;
		    c.Command << "PAUSE";
		};
	\end{verbatim}
  \caption{Hello, World!}
  \label{Hello,World}
\end{center}
\end{figure}
	产生输出:
	\begin{verbatim}
		Hello, world!
	\end{verbatim}


\tableofcontents

\listoffigures

\listoftables

\mainmatter
\part{入门篇}

	\chapter{表达式与语句}
		在本章里我们将研究两个极为重要的概念:
		能够描述一段运算的\indexed{表达式}和一个动作的\indexed{语句}。
		事实上,语句是表达式的一个超集。
		\section{变量}
			
		\section{运算}
			表达一定的数值上的计算一般被叫做\indexed{运算}。例如
			\begin{verbatim}
			1+1
			5+3
			(25-3)/2+7
			\end{verbatim}
			运算常常是\indexed{字面量}和\indexed{运算符}所进行的操作。
			
	\chapter{函数}
	\chapter{面向对象}
	
\part{进阶篇}
	\chapter{泛型}
		有的时候我们想要制作一种基于一个类的类,例如,一个整数数组。
		
	\chapter{面向集合}
	\chapter{数据结构}
	
\part{提高篇}
	\chapter{算法}
	\chapter{自订模块}
	
\renewcommand{\thepart}{}

\part{附录}
	\printindex
	
\newpage

\end{document}