% !TEX encoding = UTF-8
% !TEX program = xelatex
\documentclass[a4paper]{tufte-book}
\usepackage[CJKchecksingle]{xeCJK}
\usepackage{CJKnumb}
\usepackage{indentfirst}
\usepackage[utf8]{inputenc}
\usepackage{xunicode}
\usepackage{xltxtra}
\usepackage{listings}
\usepackage{caption}
\usepackage{tikz}
\renewcommand{\partname}{}


\usepackage{makeidx}
\makeindex


\renewcommand{\tablename}{表格}
\renewcommand{\figurename}{插图}
\renewcommand{\lstlistingname}{清单}
\renewcommand{\contentsname}{目录}
\renewcommand{\indexname}{索引}
\renewcommand{\listfigurename}{插图目录}
\renewcommand{\listtablename}{表格目录}
\renewcommand{\lstlistlistingname}{清单目录}

\makeatletter
\long\def\@makecaption#1#2{
  \vskip\abovecaptionskip
    \bfseries\footnotesize #1: #2\par
  \vskip\belowcaptionskip}
\makeatother

\begin{document}

\setlength{\parskip}{1em}
\setlength{\parindent}{2em}
\setcounter{tocdepth}{1}


\lstset{numbers=left, numberstyle=\tt \tiny,basicstyle=\tt,escapeinside=``,numberbychapter=true}


\frontmatter
\pagenumbering{roman}
\title{Hexagon~编程指南}
\author{0xHexadecimal}
\publisher{内部文档}
\maketitle

\chapter{概览}
Hexagon~是一门全新的编程语言。
它具有全新的完美类库分离语法、完全函数编程支持、强面向集合性等诸多令人喜爱的特性以及一些经过改善的语法。
一睹为快吧。
\section{Hello, world!}

将以下代码输入一个文件中,把它重命名为HelloWorld.Hexagon。
\begin{lstlisting}[caption={HelloWorld.Hexagon}]
void() Main = (){
	var c = import Hexagon::System::Console;
	c << "Hello, World!" << Hexagon::System::SpecChar::EndLine;
	c.Command << "PAUSE";
};
\end{lstlisting}
打开控制台,运行
\begin{lstlisting}
$ hgc HelloWorld.Hexagon
<编译提示>
$ hgx HelloWorld.Hexec
\end{lstlisting}
	产生输出:
\begin{lstlisting}
Hello, World!
\end{lstlisting}

\tableofcontents

\listoffigures
\addcontentsline{toc}{chapter}{\listfigurename}

\listoftables
\addcontentsline{toc}{chapter}{\listtablename}

\lstlistoflistings
\addcontentsline{toc}{chapter}{\lstlistlistingname}

\mainmatter
\pagenumbering{arabic}
\part{入门篇}
	\chapter{引入~Hexagon~: 学校成绩数据库}
		\begin{marginfigure}
		\caption{测试}
		\end{marginfigure}
		在本章中,我们将为您展示一个十分基本的程序供您参阅和研究。
		这个程序将会具有以下的功能:
		\begin{itemize}
		  \item 运行时按行读取输入
		  \item 录入学生的成绩
		  \item 查询学生的成绩
		\end{itemize}
		接下来,让我们一起来实现它吧。
		
		相关代码可在$<$待填充$>$下载。
		
	\chapter{表达式与语句}
		在本章里我们将研究两个极为重要的概念:
		能够描述一段运算的\indexed{表达式}和一个动作的\indexed{语句}。
		事实上,语句是表达式的一个超集。
		\section{变量}
			
		\section{运算}
			表达一定的数值上的计算一般被叫做\indexed{运算}。例如
			\begin{verbatim}
			1+1
			5+3
			(25-3)/2+7
			\end{verbatim}
			运算常常是\indexed{字面量}和\indexed{运算符}所进行的操作。
			
	\chapter{函数}
	\chapter{面向对象}
	
\part{进阶篇}
	\chapter{泛型}
		有的时候我们想要制作一种基于一个类的类,例如,一个整数数组。
		
	\chapter{面向集合}
	\chapter{数据结构}
	
\part{提高篇}
	\chapter{算法}
	\chapter{自订模块}
	
\renewcommand{\thepart}{}

\part{附录}

\setcounter{chapter}{0}
\renewcommand{\thechapter}{\Alph{chapter}}
\titleformat{\chapter}[display]
  {\relax\ifthenelse{\NOT\boolean{@tufte@symmetric}}{\begin{fullwidth}}{}}% format applied to label+text
  {\bfseries\fontsize{72pt}{\baselineskip}\selectfont 附录~\thechapter  \\}%
  % label
  {0pt}% horizontal separation between label and title body
  {\huge\rmfamily\itshape}% before the title body
  [\ifthenelse{\NOT\boolean{@tufte@symmetric}}{\end{fullwidth}}{}]% after the title body

	\printindex
	
\newpage

\end{document}