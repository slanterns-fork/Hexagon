% !TEX encoding = UTF-8
% !TEX program = xelatex
\documentclass[a4paper]{tufte-book}
\usepackage[CJKchecksingle]{xeCJK}
\usepackage{CJKnumb}
\usepackage{indentfirst}
\usepackage[utf8]{inputenc}
\usepackage{xunicode}
\usepackage{xltxtra}
\usepackage{listings}
\usepackage{caption}
\usepackage{tikz}
\usepackage{longtable,booktabs}
\usepackage{tabularx}
\renewcommand{\partname}{}


\usepackage{makeidx}
\makeindex


\renewcommand{\tablename}{表格}
\renewcommand{\figurename}{插图}
\renewcommand{\lstlistingname}{清单}
\renewcommand{\contentsname}{目录}
\renewcommand{\indexname}{索引}
\renewcommand{\listfigurename}{插图目录}
\renewcommand{\listtablename}{表格目录}
\renewcommand{\lstlistlistingname}{清单目录}

\makeatletter
\long\def\@makecaption#1#2{
  \vskip\abovecaptionskip
    \bfseries\footnotesize #1: #2\par
  \vskip\belowcaptionskip}
\makeatother

\renewcommand{\thefootnote}{\fnsymbol{footnote}}

\begin{document}

\setlength{\parskip}{1em}
\setlength{\parindent}{2em}
\setcounter{tocdepth}{1}


\lstset{numbers=left, numberstyle=\tt \tiny,basicstyle=\tt,escapeinside=``,numberbychapter=true}


\frontmatter
\pagenumbering{roman}
\title{Hexagon~编程指南}
\author{0xHexadecimal}
\publisher{内部文档}
\maketitle

\chapter{概览}
	Hexagon~是一门全新的编程语言。
	它具有全新的完美类库分离语法、完全函数编程支持、强面向集合性等诸多令人喜爱的特性以及一些经过改善的语法。
	一睹为快吧。
	\section{Hello, world!}
		
		将以下代码输入一个文件中,把它重命名为HelloWorld.Hexagon。
		\begin{lstlisting}[caption={你好,世界}]
Hexagon::Result() Main = (){
	var c = import Hexagon::System::Console;
	c << "Hello, World!" << Hexagon::System::SpecChar::EndLine;
	c.Command << "PAUSE";
	return Hexagon::Result.Fine;
};
		\end{lstlisting}
		打开控制台,运行
		\begin{lstlisting}
$ hgc HelloWorld.Hexagon
<编译提示>
$ hgx HelloWorld.Hexec
		\end{lstlisting}
			产生输出:
		\begin{lstlisting}
Hello, World!
		\end{lstlisting}

\tableofcontents

\listoffigures
\addcontentsline{toc}{chapter}{\listfigurename}

\listoftables
\addcontentsline{toc}{chapter}{\listtablename}

\lstlistoflistings
\addcontentsline{toc}{chapter}{\lstlistlistingname}

\mainmatter
\pagenumbering{arabic}
\part{入门篇}
	\chapter{学校成绩数据库}
		在本章中,我们将为您展示一个十分基本的程序供您参阅和研究。
		这个程序将会具有以下的功能:
		\begin{itemize}
		  \item 运行时按行读取输入
		  \item 录入学生的成绩
		  \item 查询学生的成绩
		\end{itemize}
		接下来,让我们一起来实现它吧。
		
		相关代码可在$<$待填充$>$下载。
		
		\section{程序起点}
			我们的\indexed{程序的起点}在一个叫做\indexed{Main}的\indexed{函数}里。
			\begin{lstlisting}[caption={main函数的标准形式}]
Hexagon::Result() Main; 			//声明
Main = (){
	//要执行的语句
	return Hexagon::Result.Fine;	//返回程序的运行结果
};
			\end{lstlisting}
			我们可以把一条至多条\indexed{语句}放到\emph{要执行的语句}的位置。
			如果您是第一次接触这种风格,或是第一次接触编程语言,您可以暂时不理解它们的具体含义。
			您只需将这段代码反复利用即可。
			要执行的语句就是程序的起点。
			
		\section{命令行输入/输出}
			您可以用\verb|import Hexagon::Console|来\emph{导入}所谓\indexed{控制台输入/输出流}。
			我们将这样使用它:
			\begin{lstlisting}[caption={输入/输出流}]
import Hexagon::Console;
Console << "这是我们提供的输出流。";
Console << "请输入一个整数:"
Hexagon::Interger i;
Console >> i; 	//这是输入的方式
			\end{lstlisting}
			有关于\indexed{import命令}和\indexed{输入/输出流},我们将在以后具体描述,目前您只需了解如何基本地使用它们即可满足需要。
			
		\section{语句}
			我们在执行一个动作后,用分号“\verb|;|”来结束表述,放弃现有的运算结果。
			%TODO 完成这段
			\subsection{空语句}
				如果我们直接使用一个分号,而前面没有任何表达式,这时这个语句成为了一个\indexed{空语句}。
				语法上来讲,这种做法是允许的;
				但是为了保持我们的程序尽可能的简洁,而且容易被别人读懂,请直接删除空语句,因为空语句和什么都没有其实是等效的
				\footnote{Hexagon中没有任何获取上下文语句的情况,所有的控制流都是直接传入函数对象,这一点与C/C++等语言不大相同。}。
			
		\section{基本运算}
			我们提供几个十分基础的\indexed{算术类型},它们代表一些数学意义上的实数、整数等基本情况如\ref{ls:1}所示。
			\footnotetext[2]{有关于其他进制表示我们将在以后讨论,本表一律采用十进制。}
			\footnotetext[3]{它只表示一个近似的示数。}
			\footnotetext[4]{此处表示的是,
						本类型可选一个循环节部分,写作“整数部分.小数部分.循环节”。}
			\begin{table}
			\caption{基础算术类型一览}
			\label{ls:1}
			\centering
			\begin{tabularx}{4in}{ccXXX}
				
					\toprule
						类型名 &
						中文名 &
						说明 &
						示例 &
						一般形式 \\
					\midrule
					
						Interger &
						整数型 &
						表示一个任意尺寸的整数。 &
						10, 35, 81263845618 &
						(1至9) 后接 ((0至9)重复0或多次)\footnotemark[2]
						\\
					\midrule
						
						Number &
						实数型\footnotemark[3] &
						表示一个任意尺寸的自然数。 &
						1.0, 3.14,
						1.9773.827\footnotemark[4] &
						(整数型) 后接 ((.(整数型))重复0至2次) \\
					\midrule
						
						Complex &
						复数型 & 
						表示一个由两个实数型构成的复数。 &
						i, 3.5i, 2i+1 &
						(实数型 i) 后接 (可选的(+ 实数型) \\
						
					\bottomrule
			\end{tabularx}
			\end{table}
			
			这些类型提供四则运算、乘开方、求对数等基本的数字功能,我们将在各种程序中直接使用它们。
			
	\chapter{表达式与语句}
		在本章里我们将研究两个极为重要的概念:
		能够描述一段运算的\indexed{表达式}和一个动作的\indexed{语句}。
		事实上,语句是表达式的一个超集。
		\section{变量}
			
		\section{运算}
			表达一定的数值上的计算一般被叫做\indexed{运算}。例如
			\begin{verbatim}
			1+1
			5+3
			(25-3)/2+7
			\end{verbatim}
			运算常常是\indexed{字面量}和\indexed{运算符}所进行的操作。
			
	\chapter{函数}
	\chapter{面向对象}
	
\part{进阶篇}
	\chapter{泛型}
		有的时候我们想要制作一种基于一个类的类,例如,一个整数数组。
		
	\chapter{面向集合}
	\chapter{数据结构}
	
\part{提高篇}
	\chapter{算法}
	\chapter{自订模块}
	
\renewcommand{\thepart}{}

\part{附录}

\setcounter{chapter}{0}
\renewcommand{\thechapter}{\Alph{chapter}}
\titleformat{\chapter}[display]
  {\relax\ifthenelse{\NOT\boolean{@tufte@symmetric}}{\begin{fullwidth}}{}}% format applied to label+text
  {\bfseries\fontsize{72pt}{\baselineskip}\selectfont 附录~\thechapter  \\}%
  % label
  {0pt}% horizontal separation between label and title body
  {\huge\rmfamily\itshape}% before the title body
  [\ifthenelse{\NOT\boolean{@tufte@symmetric}}{\end{fullwidth}}{}]% after the title body

	\printindex
	
\newpage

\end{document}